%-------------------------------------------------------------------------------
%	SECTION TITLE
%-------------------------------------------------------------------------------
\cvsection{Projects}


%-------------------------------------------------------------------------------
%	CONTENT
%-------------------------------------------------------------------------------
\begin{cventries}

  %---------------------------------------------------------
  \cventry
  {MongoDB, NextJS, KeystoneJS, Apollo Client, Styled Components \& Jest} % Frameworks used
  {Alchemy} % Name of the project
  {\href{https://alchemy.bond/}{alchemy.bond}} % Project URL
  {Dec 2021} % Placeholder
  {
    \begin{cvitems} % Description(s) of tasks/responsibilities
      \item {Hosts a clothing store where users can browse and search for items, manage cart, and complete checkout via Stripe}
      \item {Arranges previous orders chronologically and provides a separate page that summarizes each order}
      \item {Allows users to upload items for sale and sends item images directly to Cloudinary}
      \item {Includes an \href{https://api.alchemy.bond/}{\emph{admin UI}} to manage backend data comprised of 6 schemas}
    \end{cvitems}
  }

  %---------------------------------------------------------
  \cventry
  {MongoDB, React \& Scrapy} % Frameworks used
  {Yelp Heavy} % Name of the project
  {\href{https://yelp-heavy.herokuapp.com/}{bit.do/yelp-heavy}} % Project URL
  {Dec 2019} % Placeholder
  {
    \begin{cvitems} % Description(s) of tasks/responsibilities
      \item {Provides rankings of the restaurants in New York City, taking both positive and negative Yelp reviews into account}
      \item {Allows filtering by cuisine, review count \& price range and includes links to Google Maps vs. Yelp's proprietary map}
    \end{cvitems}
  }

  %---------------------------------------------------------
  \cventry
  {Flask, Jinja, Sentiment Analysis \& API Integration} % Frameworks used
  {Relative Horoscope} % Name of the project
  {\href{https://relative-horoscope.herokuapp.com/}{bit.do/relative-horoscope}} % Project URL
  {Sep 2019} % Placeholder
  {
    \begin{cvitems} % Description(s) of tasks/responsibilities
      \item {Scans today's horoscope and converts it into a numeric score using sentiment analysis}
      \item {Ranks each zodiac sign based on the positive sentiment of the horoscope and updates daily}
    \end{cvitems}
  }

  %---------------------------------------------------------
\end{cventries}